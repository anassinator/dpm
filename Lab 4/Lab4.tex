\documentclass[letterpaper,12pt]{article}

\begin{document}

\title{Lab 4. Localization}
\date{\today}
\author{Anass Al-Wohoush \quad \quad Malcolm William Watt \\ 260575013 \quad \quad \quad \quad \quad \quad \quad 260585950\\ Team 42}
\maketitle
\mbox{}\\

\section{Data}{
\begin{table}[h]
\centering
    \begin{tabular}{|r|r|}
    \hline
    Run \# & Theta Error \\ \hline
    1      & 4           \\ \hline
    2      & 8           \\ \hline
    3      & -7          \\ \hline
    4      & 5           \\ \hline
    5      & 20          \\ \hline
    6      & 0           \\ \hline
    7      & -15         \\ \hline
    8      & 3           \\ \hline
    9      & 10          \\ \hline
    10     & 8           \\ \hline
    \end{tabular}
    \caption {Data Collected Using the Falling Edge Method (in degrees)}
\end{table}
\clearpage
\begin{table}[h]
\centering
    \begin{tabular}{|r|r|}
    \hline
    Run \# & Theta Error \\ \hline
    1      & 1           \\ \hline
    2      & 4           \\ \hline
    3      & 5           \\ \hline
    4      & -2          \\ \hline
    5      & 12          \\ \hline
    6      & 0           \\ \hline
    7      & 3           \\ \hline
    8      & -2          \\ \hline
    9      & 2           \\ \hline
    10     & 2           \\ \hline
    \end{tabular}
    \caption {Data Collected Using the Rising Edge Method (in degrees)}
\end{table}
}

\section{Data Analysis}{
\mbox{}\\
Average falling edge error: $3.6^{\circ}$\\
The standard deviation of the falling edge: $\approx 9.069^{\circ}$\\
\mbox{}\\
Average rising edge error: $2.5^{\circ}$\\
The standard deviation of the rising edge: $\approx 3.853^{\circ}$\\
\mbox{}\\
The rising edge method performed the best. The falling edge method was much less precise. So the faults with the falling edge method is that it is looking for a distinct falling point where the points around it are very similar from the ultrasonic sensor. The fact that the difference in slope is not extremely significant may cause the ultrasonic sensor to mistake noise as a falling edge. Although we tried as hard as we could to filter this out, the minute differences between the falling edges minimum points and the points around it made this method much more problematic. With the rising edge method the sensor was searching for a high slope between any two given points, so any two given points near the critical reading point will be farther apart and therefore the rising edge is less likely to be read as a result of noise.\\
The light sensor gives a lot more accurate of a measurement because there is a more clearly defined signal from the light sensor when it passes over a line and it doesn't have to rely on its measurements being 100\% accurate. What I mean by this is that the light sensor either sees or does not see a line, the odometer can have errors as a result of it reading slightly wrong distances whereas the light sensor pretty much either works or it doesn't, so there is less unexpected error.\\
\subsection{Falling Edge}
\subsubsection{Mean}
Sum:
\begin{equation}
\sum{theta values} = (4+8-7+5+20-0-15+3+10+8) cm = 36 cm
\end{equation}
The mean is the sum divided by the number of values:
\begin{equation}
\bar{x} = \frac{36}{10} cm = 3.6 cm
\end{equation}
\subsubsection{Standard Deviation}
Difference of each data point to the mean:\\
Sample, using data point 1:\\
\begin{equation}
=(4-3.6) cm = 0.4 cm
\end{equation}
Here is a list of the results of the previous step:
\begin{center}
0.4 \\
4.4 \\
-10.6 \\
1.4 \\
16.4 \\
-3.6 \\
-18.6 \\
-0.6 \\
6.4 \\
4.4 \\
\end{center}
All values in cm.\\
Now we need to square all of the previous values.\\
Here is a list of the results:
\begin{center}
0.16 \\
19.36 \\
112.36 \\
1.96 \\
268.96 \\
12.96 \\
345.96 \\
0.36 \\
40.96 \\
19.36 \\
\end{center}
All values in $cm^{2}$.\\
Now I do the sum of the values from the previous step.
\begin{center}
= 822.4 $cm^{2}$
\end{center}
Divided by 10 (the number of values):
\begin{center}
= 82.24 $cm^{2}$
\end{center}
And take the square root: 
\begin{center}
= $\sqrt{82.24} cm$\\
$\approx 9.069 cm$
\end{center}
So our standard deviation for falling edge is 9.077 cm.

\subsection{Rising Edge}
\subsubsection{Mean}
Sum:
\begin{equation}
\sum{theta values} = (1+4+5-2+12-0+3-2+2+2) cm = 25 cm
\end{equation}
The mean is the sum divided by the number of values:
\begin{equation}
\bar{x} = \frac{25}{10} cm = 2.5 cm
\end{equation}
\subsubsection{Standard Deviation}
Difference of each data point to the mean:\\
Sample, using data point 1:\\
\begin{equation}
=(1-2.5) cm = -1.5 cm
\end{equation}
Here is a list of the results of the previous step:
\begin{center}
-1.5 \\
1.5 \\
2.5 \\
-4.5 \\
9.5 \\
-2.5 \\
0.5 \\
-4.5 \\
-0.5 \\
-0.5 \\
\end{center}
All values in cm.\\
Now we need to square all of the previous values.\\
Here is a list of the results:
\begin{center}
2.25 \\
2.25 \\
6.25 \\
20.25 \\
90.25 \\
6.25 \\
0.25 \\
20.25 \\
0.25 \\
0.25 \\
\end{center}
All values in $cm^{2}$.\\
Now I do the sum of the values from the previous step.
\begin{center}
= 148.5 $cm^{2}$
\end{center}
Divided by 10 (the number of values):
\begin{center}
= 14.85 $cm^{2}$
\end{center}
And take the square root: 
\begin{center}
= $\sqrt{14.85} cm$\\
$\approx 3.847 cm$
\end{center}
So our standard deviation for falling edge is 3.847 cm.
}

\section{Observations and Conclusion}{
One could easily determine the initial position of the robot by simply considering the minimum values returned by the ultrasonic sensor since these values should correspond to the distance between the robot and the wall when the sensor is perpendicular to the wall. Doing this on both walls would allow us to approximate where the $270^{\circ}$ and $180^{\circ}$ are and estimate our heading accordingly. This is slightly inaccurate however due to noise and to the sensor's range which is limited when trying to measure shorter distances.
}

\section{Further Improvements}{
A clipping filter may eliminate random noise relatively well, but a better method would be a moving average filter which averages the past ten or so values over time. This would essentially work as a low pass filter by filtering out any sudden bloops in the data stream while preserving the relatively correct signal whereas the clipping filter essentially does nothing but saturate the signal and loses information. This is also bound to be more accurate due to also eliminating the noise under the clipping threshold.\\

The ultrasonic sensor is inaccurate due to depending on the reflectivity of the object, to being subject to echoes and to not being able to distinguish the angle of incidence. A more accurate and reliable sensor would be a laser range finder which measures the time travel of a narrow laser beam to estimate the distance. This is more reliable due to its much higher polling rate and narrower cone of view. And if one wants to be really fancy, we could always invest in a LiDAR.\\

The rising and falling edge methods only work when the robot is precisely on the diagonal of the square. A more reliable method is hence peak detection where one detects the peaks of a signal instead of triggering on its edges. With this we could detect the minima and hence detect when the robot is perpendicular to the walls it faces no matter how far off the diagonal it is within the sensor's range.

}

\end{document}
